\chapter{Estado del Arte}

La implementación de herramientas tecnologicas en actividades de la vida cotidiana es cada vez mas evidente, actualmente contamos con asistentes virtuales como Alexa por parte de amazon, google assistant desarrolado por google, microsoft tiene por su parte a cortana, y Siri de la mano de Apple, que facilitan tareas como programar recordatorios, repdorucir musica, buscar información en internet, realizar llamadas o enviar correos, todo esto mediante comandos de voz realizados por el usuario.

\citet{z} nos enseña que la interación entre humanos y chatbots se convertira gradualmente en parte de la vida cotidiada de nuestra sociedad. De a cuerdo a las necesidades que se presentan, las insituciónes buscan como llegar a sus usuarios de manera mas eficiente, es el caso de “Esperanza” un Chatbot desarrollado por la cadena de televisión Novo Tempo, perteneciente a
IASD, esto como un recurso para bridar apoyo a los estudiantes de la Escuela Bíblica Digital\citep{476172132004}.

En 2020 tras la llegada del Covid-19 nuestra sociedad se vio obligada a adaptarse e implementar nuevas estrategias, tomando como apoyo las nuevas tecnologias, es el ccaso del Chatbot desarrollado por en la Ciudad de cali, Colombia, el cual permitia brindar información acerca de las nuevas noticias respecto a la situcaión por pandemia, responde preguntas generales y también perguntas relaciónadas con el numero de casos y puntos geograficos en la ciudad de Cali\citep{kali}.

Tanto en el caso del prototipo desarrollado por la ciudad de cali, como en el chatbot "Esperanza", la necesidad era la misma, brindar información en tiempos de pandemia para evitar la interación directa con otro ser humano, aplicando herramientas de NLP en chatbots para que los usuarios tuvieran una aceptación positiva de estas herramientas, y en donde la implementación no necesitara una capacitación previa, pues se entiende que el usuario es la persona promedio, por lo que la herramienta debe ser intutitiva y adpatarse al tipo de información que se desea brindar, asi como poder actualizarse y adaptarse a las nuevas necesidades.

El uso de Chatbots como herramienta no tiene por que limitarse a un sector en particular y es lo que hace de esta herramienta un fuerte aliado cuando a brindar información se refiere, es el caso del Centro de Investigaciónes y Estudios Turisticos, Argentina, quienes en 2020 decidieron implementar el Chatbot Kayak, para que mediante un experimento con 102 alumnos de la carrera de turismo se midiese la utilidad de un chatbot para brindar información en el sector de viajes y turismo, estos interactiaban libremente con el chatbot por 10 minutos, y posteriomente se les hacia una encuesta con escala Likert de 7 puntos, para evaluar la calidad de la experiencia, dando como resultado una experiencia agradable pero con posibles aspectos que necesitan mejora\citep{turismo}.

Para \citet{publicservices} la implemetación de IA en el sector publico es evidente, pues las tecnologias digitales han cambiado y seguiran cambiando rapidamente el panorama que se tiene en la presentación de servicios publicos, incluso ya para el 2020 se sabia que el modo en que la tecnologia esta avanzando de manera tan vertiginoza, en donde tecnologias como el big data, machine learning, NLP, etc. han propiciado que la aplicación de la IA sea cada vez mas necesario para mantenerse a la par de esta revolución tecnologica, no solo para sectores directamente relaciónados como la computación, o los sistemas computaciónales, sino tambien en el ambito social, educativo, publico y privado, casi para cualquier sector de la sociedad la implementación de las nuevas tecnologias supone un gran apoyo en las tareas cotidianas de estos. Y aunque esto supone muchos desafios, es inegable que cada vez se convierte mas en una necesidad que en un lujo la adaptación a las nuevas tecnologias\citep{publico}

De este modo \citet{aiftps} afirma que cada vez mas organizaciónes del sector publico estan interesadas en el uso e implementación de IA parala ciencia de datos. Aunque se debe tomar en cuenta que no todas las personas estan capacitadas para afrontar los desafios que exige esta nueva transformación, tal como afirma \citet{agarwal2018public}, ademas de la predisposición adoptada por los usuarios finales a los que se pretende llegar con estas nuevas tecnologias, se debe tener encuenta que una exageración en la implementación de IA da com resultado una mayor incertidumbre en cuan positivo podria resultar para el sector privado, tal como lo señala \citet{sun2019mapping}

En un espacio academico la situación es similar a la implementación de las nuevas tecnologias en el sector publico, no debemos olvidar que apesar de los multiples usos y ventajas que nos proporciona el convinar metodologias innovadoras con el uso de tecnologias novedosas las cuales dan como resultado profesionales creativos y con multiples habilidades \citep{renz2020prerequisites} esta se introduce de manera lenta, a fin de volverse parte de nuestra vida, \citet{331464460002} menciona que la tecnologia ah formado completamente parte de nuestras vidas, cuando no percibimos que existe, como ejemplo, el uso de la electricidad que forma parte de nuestro día a día y sin embargo naide se da cuenta de la importancia que esta tiene, como lo fue en el momento de su invención. Esa es la meta aspiracional que se pretende alcanzar con el uso de las teconologias avanzadas frente al reto de la innovación educativa.

De este modo, retomando la propuesta del uso de tecnologias cognitivas, \citet{asistVirAcad} mencionan que las nuevas tecnologias en conjunto con el gran avanze y presencia que tiene el internet actualmente proporcionan nuevas oportunidades al sistema educativo para mejorar los sistemas de información para los estudiantes, en donde proponen el uso de un asistente virtual academico aplicando tecnologias cognitivas de procesamiento natural. En este trabajo se hace una comparativa entre las principales plataformas de procesamiento de lenguaje natural como lo son, Dialogflow, Rasa, Amazon Lex y IBM Watson, poniendo especial atencion en aspectos como los costros, conocimientos que son requeridos para implementar dichos sistemas, integraciones con mensajerias ya existentes, etc.

En cuanto a la existencia de Chatbots en el mercado, se pueden encontrar distintos, como lo es Chatterbot, el cual es una biblioteca de python con la finalidad de facilitar la generación de respuestas ante la entrada de datos por parte de un usuario\cite{chatterbot}

Hasta aqui, se ha resaltado la importancia de los Chatbots, asi como algunos ejemplos de ellos, y sus usos en distintos sectores sociales, por lo que lo siguiente seria señalar aquellos trabajos que buscan clasificar los chatbots en funcion de diferentes parametros, de acuerdo con \citet{adamopoulou2020overview} los parametros a tomar en cuenta son los siguientes:

\begin{itemize} 
	\item Dominio de conocimiento
	\item Servicio prestado
	\item Objetivos
	\item Metodo de procesamiento de entrada y generación de respuestas
	\item Ayuda humana
	\item Método de construcción
\end{itemize}

Mientras que otros como \citet{lokman2018modern} secciona la clasificación de los Chatbots en dos: Diseño arquitectonico y proceso de implementación. En donde englobal aspectos como la generación de respuestas, el procesamiento de textos y modelo de aprendiaje en el Diseño arquitectonico y el uso de datos para la parte del proceso de implementación.

Estos parametros nos sirven para lograr una clasificación adecuada de los chatbots, sin embargo para medir la efectividad de estos los parametros a tomar en cuenta son distintos. Pues mientras que en los bots orientados a tareas el rendimiento se mide mediante el exito de la tarea, en los chatbots sociales las medición del rendimiento resulta mas complicada\citep{shawar2007different}. Para tal caso \citet{zhou2020design} afronta este problema evaluando el CPS\footnote{Conversation-turns Per Session} y NAU\footnote{Number of Active Users} teniendo como objetivo mantener el interes del usuario en la conversación con el Chatbot de Microsoft XiaoIce.

En otro caso \citet{sedoc-etal-2019-chateval} proponen el uso de ChatEval, que es una herramienta que permite llevar a cabo una evaluación de Chatbots.

Finalmente \citet{cui2017superagent} con superagent, el cual es un chatbot de servicio al cliente para sitios web de comercio electronico dice que los Chatbots de este tipo porporcionan utilidad tanto para los usuarios, como para el personal a cargo de atender a los usuarios, reduciendo costos y complementando a la web convencional.

Cuando se habla de estos avanzes tecnologicos no se puede dejar de lado las tecnicas y el conocimiento necesarios en la aplicación de estas tecnologias, la cociedad Mexicana de inteligencia artifical explica que la ciencia encargada del procesamiento automatico del lenguaje natural es llamado lingüística computacional\citep{gelbukh2018procesamiento}.

El procesamiento de lenguaje natural tiene la capacidad de analizar grandes corpus de información y detectar similitudes, asi como clasificar esta información\cite{sancho2020aplicacion,miranda2022mention} por ello el potencial que adquiere en trabajos con gran cantidad de información esta muy por encima de los metodos usados hasta hace unos años.

Para este trabajo de investigación la herramienta principal que se pretende usar es la libreria NLTK de python por lo cual es importante la consulta de trabajos donde este presenta dicha herramienta. En su caso \citet{schmitt2019replicable} hace una comparativa entre StanfordNLP, NLTK, OpenNLP, SpaCy y Gate los cuales a la fecha son los software mas conocidos para uso generico, en su caso NLTK ofrece una amplia gama de bibliotecas y herramientas para el NLP, a diferencia de otros como Spacy cuya caracteristicas principal es su rapidez.

La mayoria de articulos mencionados, hacen referencia a chatbots, pero cuando se busca información hacerca del NLP o en un caso mas especifico, de la herramienta de python NLTK se puede distinguir que la mayoria de articulos y temas encontrados son acerca del uso de esta herramienta en grandes cantidades de texto, asi como en la deteccion de sentimientos u emociones de los usuarios. Esto supone a fin de cuentas una ventaja para la tarea que se quiere realizar, pues a fin de poder medir la eficacia de esta herramienta contamos con variables tales como la satisfaccion del usuario, el tiempo de uso, la velocidad de respues, etc. A fin de minimizar el trabajo humano requerido\citep{wang2021application}.