\chapter{Estado del Arte}

La implementación de herramientas tecnologicas en actividades de la vida cotidiana es cada vez mas evidente, actualmente contamos con asistentes virtuales como Alexa por parte de amazon, google assistant desarrolado por google, microsoft tiene por su parte a cortana, y Siri de la mano de Apple, que facilitan tareas como programar recordatorios, repdorucir musica, buscar información en internet, realizar llamadas o enviar correos, todo esto mediante comandos de voz realizados por el usuario.

\citet{z} nos enseña que la interación entre humanos y chatbots se convertira gradualmente en parte de la vida cotidiada de nuestra sociedad. De a cuerdo a las necesidades que se presentan, las insituciónes buscan como llegar a sus usuarios de manera mas eficiente, es el caso de “Esperanza” un Chatbot desarrollado por la cadena de televisión Novo Tempo, perteneciente a
IASD, esto como un recurso para bridar apoyo a los estudiantes de la Escuela Bíblica Digital\citep{476172132004}.

En 2020 tras la llegada del Covid-19 nuestra sociedad se vio obligada a adaptarse e implementar nuevas estrategias, tomando como apoyo las nuevas tecnologias, es el ccaso del Chatbot desarrollado por en la Ciudad de cali, Colombia, el cual permitia brindar información acerca de las nuevas noticias respecto a la situcaión por pandemia, responde preguntas generales y también perguntas relaciónadas con el numero de casos y puntos geograficos en la ciudad de Cali\citep{kali}.

Tanto en el caso del prototipo desarrollado por la ciudad de cali, como en el chatbot "Esperanza", la necesidad era la misma, brindar información en tiempos de pandemia para evitar la interación directa con otro ser humano, aplicando herramientas de NLP en chatbots para que los usuarios tuvieran una aceptación positiva de estas herramientas, y en donde la implementación no necesitara una capacitación previa, pues se entiende que el usuario es la persona promedio, por lo que la herramienta debe ser intutitiva y adpatarse al tipo de información que se desea brindar, asi como poder actualizarse y adaptarse a las nuevas necesidades.

El uso de Chatbots como herramienta no tiene por que limitarse a un sector en particular y es lo que hace de esta herramienta un fuerte aliado cuando a brindar información se refiere, es el caso del Centro de Investigaciónes y Estudios Turisticos, Argentina, quienes en 2020 decidieron implementar el Chatbot Kayak, para que mediante un experimento con 102 alumnos de la carrera de turismo se midiese la utilidad de un chatbot para brindar información en el sector de viajes y turismo, estos interactiaban libremente con el chatbot por 10 minutos, y posteriomente se les hacia una encuesta con escala Likert de 7 puntos, para evaluar la calidad de la experiencia, dando como resultado una experiencia agradable pero con posibles aspectos que necesitan mejora\citep{turismo}.

Para \citet{publicservices} la implemetación de IA en el sector publico es evidente, pues las tecnologias digitales han cambiado y seguiran cambiando rapidamente el panorama que se tiene en la presentación de servicios publicos, incluso ya para el 2020 se sabia que el modo en que la tecnologia esta avanzando de manera tan vertiginoza, en donde tecnologias como el big data, machine learning, NLP, etc. han propiciado que la aplicación de la IA sea cada vez mas necesario para mantenerse a la par de esta revolución tecnologica, no solo para sectores directamente relaciónados como la computación, o los sistemas computaciónales, sino tambien en el ambito social, educativo, publico y privado, casi para cualquier sector de la sociedad la implementación de las nuevas tecnologias supone un gran apoyo en las tareas cotidianas de estos. Y aunque esto supone muchos desafios, es inegable que cada vez se convierte mas en una necesidad que en un lujo la adaptación a las nuevas tecnologias\citep{publico}

De este modo \citet{aiftps}