\chapter{Estado del Arte}

La implementación de herramientas tecnologicas en actividades de la vida cotidiana es cada vez mas evidente, actualmente contamos con asistentes virtuales como Alexa por parte de amazon, google assistant desarrolado por google, microsoft tiene por su parte a cortana, y Siri de la mano de Apple, que facilitan tareas como programar recordatorios, repdorucir musica, buscar información en internet, realizar llamadas o enviar correos, todo esto mediante comandos de voz realizados por el usuario.

\citet{z} nos enseña que la interación entre humanos y chatbots se convertira gradualmente en parte de la vida cotidiada de nuestra sociedad. De a cuerdo a las necesidades que se presentan, las insituciónes buscan como llegar a sus usuarios de manera mas eficiente, es el caso de “Esperanza” un Chatbot desarrollado por la cadena de televisión Novo Tempo, perteneciente a
IASD, esto como un recurso para bridar apoyo a los estudiantes de la Escuela Bíblica Digital\citep{476172132004}.
