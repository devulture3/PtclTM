 \chapter{Planteamiento del problema de investigación}

Hoy en día el uso de tecnologías y medios de comunicación digitales resulta indispensable durante el envio y recepción de información. Para las nuevas generaciones es difícil imaginar una situación en la que no baste con tan solo un clic para comunicarse con alguien en cualquier horario y lugar, esto sumado a la creciente necesidad a lo inmeadiato, genera una problemática para aquellas organizaciones acostumbradas a trasmitir información mediante los métodos mas comunes, como avisos o atendiendo directamente a ventanilla de manera presencial. 

\section{Antecedentes}

Cuando se habla de esto, no se pueden ignorar aspectos como la sociedad de la información y la cultura digital, estos no son conceptos nuevos pues la sociedad de la información es una “sociedad que crece y se desarrolla alrededor de la información y aporta un florecimiento general de la creatividad intelectual humana, en lugar de un aumento del consumo material”\citep[págs. 124]{bib1}. y en donde se entiende por Cultura Digital “una forma de relaciones entre personas, con mediación tecnológica, que se diferencia de la cultura análoga y de la manera más tradicional de comunicarnos”\citep{art1}.
De modo que, el no aprovechar los recursos tecnológicos que se tienen a la mano para apoyar los métodos tradicionales en las consultas de información deja a la deriva a cualquier institución u organización que no esté predispuesta a dicho cambio.

Según datos de la INEGI en México hay 84.1 millones de usuarios de internet y 88.2 millones de usuarios de teléfonos celulares\citep{inegi2021}, dado que estos datos son del año 2020, muy probablemente, esta cifra aumento con respecto al 2022, es decir más del 72.0\% de la población de seis años o más, cuenta con acceso al internet, por lo que para el envío y recepción de información esto supone una gran ventaja sobre los métodos de consultas tradicionales, sobre los cuales la mayoría de las instituciones hace uso.

Espacios Universitario como lo son la UAEM hacen uso de las TIC's en procesos de incorporación para nuevos alumnos, registro y manejo de los datos del alumnado, etc. tramites, etc. para la difusion de información la mayoria de universidades de México cuenta con plataformas digitales, paginas de Facebook, etc. Y aun haciendo uso de las TIC's en todo lo anteriormente señalado, gran parte de los espacios academicos solo hacen uso de las TIc's para difundir la información de la misma manera que se a hecho desde que se incorporo la tecnologias, es decir, la evoluciñon y adaptabilidad que tienen las instituciñones es minima con respecto a los avaces tecnologicos de los ultimos años.

El Centro Universitario UAEM Atlacomulco, incorpora cada año nuevos alumnos, en cada una de las carreras que oferta, y cada semestre los alumnos realizan tramites para becas, servicio social, practicas profesionales, entre otros tramites que tienen que ver con el area de Asuntos Estudiantiles. Existen diversas formas de atender las solicitudes que se presentan, pues Asuntos Estudiantiles cuenta con correo electronico, pagina de Facebook, y un espacio fisico al cual acudir, en un horario de 09:00 a 17:00 de Lunes a Viernes, aunque este horario puede variar dependiendo de la disponibilidad de tiempo de la persona a cargo del area. Aparentenemte parece inexitente un problema en este caso, incluso viendolo de manera objetiva, pues el flujo de alumnos año con año es constante, es decir no aumentan ni disminuyen considerablemente. El cambio que puede observarse y debe ser tomado en cuenta es la manera en la que las nuevas generaciones estan familiarizadas con la tecnologia, y en el como, de no empezar a implementar nuevas estrategias para mejorar aspectos tan cotidianos como las consultas de información, deja un paso atras a las instituciónes con respecto a aquellas que se han adaptado a estos cambios.

No es sorpresa que en el año 2020 con la llegada del COVID-19 a México, en donde las TIC's tuvieron un papel fundamental para la educación, muchas instituciónes tuviesen que adaptar muchos procesos a un modelo mas digital, y que al dia de hoy, en el año 2023, despues de regresar a nuestra modalidad presencial, muchos de estos procesos se hayan quedado, pues resultaron mas eficientes que como anteriormente se habia manejado.

\section{Definición del problema}

La propuesta de implementar un CHATBOT surge tras la necesidad de optimizar algunos procesos en el Area de asuntos estudiantiles, pretende servir como apoyo para atender y procesar solicitudes generales, dando al personal de dicha area mayor tiempo para realizar tareas que requieran de una atención mas precisa, sirviendo asi como apoyo para el personal, y como una herramienta en la cual los alumnos puedan resolver dudas, aun en un horario no laboral, sin gastar tiempo ni dinero en trasladarse de un lugar a otro.

Algunas de las caracteristicas con las que cuenta un chatbot son las siguientes:
\begin{itemize}
	\item Cuenta con cierta autonomia, lo cual permite atender y procesar solicitudes sin la necesidad de una persona fisica que lo controle
	\item Alta disponibilidad, tanto de tiempo como para procesar diferentes solicitudes a la vez
	\item Facilitan la interación con los usuarios
\end{itemize}

Todas estas caracteristicas, hacen optimo el uso de Chatbots para antender solicitudes de información, especialmente si esta es recurrente. Por otro lado, los Chatbots requieren una preparación inicial, en donde se le instruye para poder atender de manera adecuada, este entrenamiento debe ser contaste y actualizarse dependiendo del tipo de solicitud que se haga. Actualmente los CHatbots hacen uso de lenguaje natural, asi como de Machine Learning, para procesar solcitudes de manera mas natural con el susuario y dar respuestas lo mas parecidas a como lo haria una persona normal.

\begin{table}[!ht]
\centering
\caption{La siguiente tabla muestra algunas de las psoibles causas y consecuencias de nuestro problema. Adaptado de \citep{bib_tesis1}.} 
\label{tab:ejemplo}
\begin{tabular}{|p{5.0cm}|p{9.0cm}|}
\hline
Causas & Consecuencias \\ [0.5ex] 
\hline\hline & \\
Solcitudes Repetidas & El personal pierde tiempo atentiendo a las mismas preguntas \\ [1.5ex]
Tiempo de atención limitado & Los estudiantes solo pueden atender sus dudas en un horario especifico \\[1.5ex]
Gastos economicos inecesarios & Los estudiantes gastan dinero en el traslado a la institución para antender dudas \\[1.5ex]
Fuerte carga laboral en periodos & Existe una sobrecarga de trabajo en periodos como las periodos de becas, trmites de practicas y servicio social, que provoca un agotamiento laboral adicional \\ [1.5ex]
\hline
\end{tabular}
\end{table}
\clearpage
\section{Objetivos de la investigación}

\subsubsection{Objetivo general}
Implementar un Chatbot con NLP\footnote{NLP de Natural Lenguage Procesing, o procesamiento del lenguaje natural, de aqui en adelante se abreviara como NLP} para mejorar las consultas de información recurrente en el area de asuntos estudiantiles del CU UAEM Atlacomulco

\subsubsection{Objetivos especificos}

\begin{itemize}
	\item Identificar como el NLP ayuda en la mejora del uso de las TIC's hoy en día.
	\item Investigar el estado actual de satisfacción de los alumnos con respecto a los metodos tradicionales para consultas de información en el CU UAEM Atlacomulco.
	\item Diseñar un Chatbot con NLP capaz de atender solicitudes a preguntas generales para el Area de suntos estudiantiles del CU UAEM Atlacomulco
\end{itemize} 

\section{Preguntas de investigación}

\subsubsection{Pregunta central}

¿Como implementar Procesamiento de Lenguaje natural en Chatbots mejorara el servicio de Asuntos Estudiantiles para los alumnos del Centro Universitario UAEM Atlacomulco?

\subsubsection{Preguntas secundarias}

\begin{itemize}
	\item ¿Cuáles son las aplicaciónes actuales del NLP, y en que medida ayudan a mejorar las TIC´s hoy en día?
	\item ¿Qué opinion tienen los alumnos sobre implementar nuevas estrategias tecnologicas para antender dudas generales en el area de Asuntos Estudiantiles?
	\item ¿Que requerimentos se necesitan para implementar un ChatBot con NLP y en que tiempo puede comenzar a usarse?
\end{itemize} 

\section{Justificación}

Con la implementación de este CHATBOT de pretende brindar apoyo a la comunidad estudiantil de CU UAEM Atlacomlco, asi como al departamento de asuntos estudiantiles, para poder brindar un servicio que se adapte a las necesidades actuales y futuras que pueda tener la institución, asi mismo, pretende mostrar las capacidades o posibles aplicaciónes que el uso de este tipo de tecnologias propone para ayudar a diferentes areas posteriormente, como por ejemplo, para brindar información por parte de control escolar a posibles nuevos estudiantes sobre porcesos de inscripción, brindar ayuda a nuevos estudiantes sobre el uso de pataformas como el sistema de control escolar UAEMex, brindar ayuda sobre procesos de reinscripción, pagó de reposición, constancias de estudio, etc. Chatbot no sustituye a las figuras docentes ni a las de personal de administración y servicios, sino que sustituye algunas de las tareas que asumen estas figuras, las complementa y las ayuda\citep{art2}.

\section{Requerimentos del Proyecto}

En este apartado debe explicarse el costo que la investigación requiere (monetario, en
recursos materiales o humanos, o en tiempo), y comparar este costo con los beneficios
adquiridos (conocimiento, tecnología, beneficio social, nuevas líneas de investigación,
potencial de retorno económico del proyecto).
También se debe describir en este apartado el por qué se propone el uso de determinadas
herramientas, técnicas, algoritmos, plataformas de desarrollo, etc., cuando éstas se
identificado durante el desarrollo del protocolo.

\section{Impactos}
\begin{itemize} 
	\item Tecnológico: Se pretende buscar un metodo en el cual se pueda implementar de manera eficiente el Procesamiento de Lenguaje Natural mediante un Chatbot, de modo que el impacto econocmico sea minimo, implementando librerias en python, y haciendo uso de aplicación de mensajeria instantanea como telegram o Whatsapp.
	\item Económico: Ahorro en costos de traslado por parte de los alumnos al no tener que asistir de manera presencial al espacio academico.
	\item Social: Se pretende tener un impacto positivo en nuevas generaciónes, apoyando en los procesos academicos, mejorando la calidad en la atención al alumnado.
\end{itemize}
