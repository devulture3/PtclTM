\chapter{Planteamiento del problema de investigación}

Hoy en día el uso de tecnologías y medios de comunicación digitales resulta indispensable durante el envio y recepción de información. Para las nuevas generaciones es difícil imaginar una situación en la que no baste con tan solo un clic para comunicarse con alguien en cualquier horario y lugar, esto sumado a la creciente necesidad a lo inmeadiato, genera una problemática para aquellas organizaciones acostumbradas a trasmitir información mediante los métodos mas comunes, como avisos o atendiendo directamente a ventanilla de manera presencial. 

\section{Antecedentes}

Cuando se habla de esto, no se pueden ignorar aspectos como la sociedad de la información y la cultura digital, estos no son conceptos nuevos pues la sociedad de la información es una “sociedad que crece y se desarrolla alrededor de la información y aporta un florecimiento general de la creatividad intelectual humana, en lugar de un aumento del consumo material”. (Masuda, 1984:124) y en donde se entiende por Cultura Digital “una forma de relaciones entre personas, con mediación tecnológica, que se diferencia de la cultura análoga y de la manera más tradicional de comunicarnos”. (Ministerio de Cultura de la República de Colombia, 2010).
De modo que, el no aprovechar los recursos tecnológicos que se tienen a la mano para apoyar los métodos tradicionales en las consultas de información deja a la deriva a cualquier institución u organización que no esté predispuesta a dicho cambio.
Según datos de la INEGI en México hay 84.1 millones de usuarios de internet y 88.2 millones de usuarios de teléfonos celulares(Insertar cita) endutih 2020, dado que estos datos son del año 2020, muy probablemente, esta cifra aumento con respecto al 2022, es decir más del 72.0\% de la población de seis años o más, cuenta con acceso al internet, por lo que para el envío y recepción de información esto supone una gran ventaja sobre los métodos de consultas tradicionales, sobre los cuales la mayoría de las instituciones hace uso.


\section{Definición del problema}

Desarrollar el enunciado del problema, de la manera más clara y breve posible. También es conveniente delimitar el problema, es decir, describir el alcance del proyecto, de tal modo que se indique hasta qué punto se desarrollará el trabajo de investigación. La definición del problema debe dar cuenta de por lo menos uno de los siguientes aspectos:

\begin{itemize}
	\item La brecha que existe entre el conocimiento actual y el nuevo, que se obtendrá al concluir la investigación
	\item La diferencia entre la situación que se tiene antes de solucionar el problema, y la que se pretende alcanzar mediante la implementación de la solución.
\end{itemize}

\section{Preguntas de investigación}

\subsubsection{Pregunta central}

Identifique una pregunta central, sobre la que versa el propósito de este tema (recuerde que un tema contiene: línea de investigación y propósito del estudio). Escriba la pregunta aquí. 

\subsubsection{Preguntas secundarias}

Identifique, posiblemente mediante una lista o lista enumerada, algunas interrogantes secundarias sobre el tema. Estas pueden ir cambiando, aumentando o eliminándose conforme el tema se delimite y se avance en la indagación documental. 

Especificar las interrogantes que se pretenden resolver con el desarrollo del proyecto de investigación. Las preguntas pueden ser tanto de índole teórico (conocimiento adquirido) como tecnológico (aplicación de la ciencia) \citep{Behar_2008_Metodologia}. Las preguntas de investigación se identifican con interrogantes como las siguientes:

\begin{itemize}
	\item ¿Qué se requiere realizar y para qué?
	\item ¿Qué relación existe entre el problema planteado y el conocimiento científico a desarrollar?
	\item ¿Qué brechas científicas o tecnológicas se pretenden acortar?
	\item ¿De qué modo se relacionan algunas variables?
	\item	¿De qué manera se miden o cuantifican algunas magnitudes o variables?
	\item ¿Cuáles son las variables físicas involucradas en la solución del problema? 
	\item ¿Qué alternativas de solución existen? ¿Cuál es la mejor? 
	\item ¿Qué nuevas formas de solución se están planteando?
\end{itemize}