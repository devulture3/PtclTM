\chapter{Planteamiento del problema de investigación}

Hoy en día el uso de tecnologías y medios de comunicación digitales resulta indispensable durante el envio y recepción de información. Para las nuevas generaciones es difícil imaginar una situación en la que no baste con tan solo un clic para comunicarse con alguien en cualquier horario y lugar, esto sumado a la creciente necesidad a lo inmeadiato, genera una problemática para aquellas organizaciones acostumbradas a trasmitir información mediante los métodos mas comunes, como avisos o atendiendo directamente a ventanilla de manera presencial. 

\section{Antecedentes}

Cuando se habla de esto, no se pueden ignorar aspectos como la sociedad de la información y la cultura digital, estos no son conceptos nuevos pues la sociedad de la información es una “sociedad que crece y se desarrolla alrededor de la información y aporta un florecimiento general de la creatividad intelectual humana, en lugar de un aumento del consumo material”\citep[págs. 124]{bib1}. (Masuda, 1984:124) y en donde se entiende por Cultura Digital “una forma de relaciones entre personas, con mediación tecnológica, que se diferencia de la cultura análoga y de la manera más tradicional de comunicarnos”. (Ministerio de Cultura de la República de Colombia, 2010).
De modo que, el no aprovechar los recursos tecnológicos que se tienen a la mano para apoyar los métodos tradicionales en las consultas de información deja a la deriva a cualquier institución u organización que no esté predispuesta a dicho cambio.
Según datos de la INEGI en México hay 84.1 millones de usuarios de internet y 88.2 millones de usuarios de teléfonos celulares(Insertar cita) endutih 2020, dado que estos datos son del año 2020, muy probablemente, esta cifra aumento con respecto al 2022, es decir más del 72.0\% de la población de seis años o más, cuenta con acceso al internet, por lo que para el envío y recepción de información esto supone una gran ventaja sobre los métodos de consultas tradicionales, sobre los cuales la mayoría de las instituciones hace uso.
